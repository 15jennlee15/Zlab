\documentclass[man]{apa6}
\usepackage{lmodern}
\usepackage{amssymb,amsmath}
\usepackage{ifxetex,ifluatex}
\usepackage{fixltx2e} % provides \textsubscript
\ifnum 0\ifxetex 1\fi\ifluatex 1\fi=0 % if pdftex
  \usepackage[T1]{fontenc}
  \usepackage[utf8]{inputenc}
\else % if luatex or xelatex
  \ifxetex
    \usepackage{mathspec}
  \else
    \usepackage{fontspec}
  \fi
  \defaultfontfeatures{Ligatures=TeX,Scale=MatchLowercase}
\fi
% use upquote if available, for straight quotes in verbatim environments
\IfFileExists{upquote.sty}{\usepackage{upquote}}{}
% use microtype if available
\IfFileExists{microtype.sty}{%
\usepackage{microtype}
\UseMicrotypeSet[protrusion]{basicmath} % disable protrusion for tt fonts
}{}
\usepackage{hyperref}
\hypersetup{unicode=true,
            pdftitle={Maternal Emotion Dysregulation and its Association with Child Internalizing and Externalizing Behaviors and Heart Rate Variability},
            pdfauthor={Jackie O'Brien, Jenn Lewis, \& Yoel Everett},
            pdfkeywords={emotion regulation, parenting, child outcomes},
            pdfborder={0 0 0},
            breaklinks=true}
\urlstyle{same}  % don't use monospace font for urls
\usepackage{graphicx,grffile}
\makeatletter
\def\maxwidth{\ifdim\Gin@nat@width>\linewidth\linewidth\else\Gin@nat@width\fi}
\def\maxheight{\ifdim\Gin@nat@height>\textheight\textheight\else\Gin@nat@height\fi}
\makeatother
% Scale images if necessary, so that they will not overflow the page
% margins by default, and it is still possible to overwrite the defaults
% using explicit options in \includegraphics[width, height, ...]{}
\setkeys{Gin}{width=\maxwidth,height=\maxheight,keepaspectratio}
\IfFileExists{parskip.sty}{%
\usepackage{parskip}
}{% else
\setlength{\parindent}{0pt}
\setlength{\parskip}{6pt plus 2pt minus 1pt}
}
\setlength{\emergencystretch}{3em}  % prevent overfull lines
\providecommand{\tightlist}{%
  \setlength{\itemsep}{0pt}\setlength{\parskip}{0pt}}
\setcounter{secnumdepth}{0}
% Redefines (sub)paragraphs to behave more like sections
\ifx\paragraph\undefined\else
\let\oldparagraph\paragraph
\renewcommand{\paragraph}[1]{\oldparagraph{#1}\mbox{}}
\fi
\ifx\subparagraph\undefined\else
\let\oldsubparagraph\subparagraph
\renewcommand{\subparagraph}[1]{\oldsubparagraph{#1}\mbox{}}
\fi

%%% Use protect on footnotes to avoid problems with footnotes in titles
\let\rmarkdownfootnote\footnote%
\def\footnote{\protect\rmarkdownfootnote}


  \title{Maternal Emotion Dysregulation and its Association with Child
Internalizing and Externalizing Behaviors and Heart Rate Variability}
    \author{Jackie O'Brien\textsuperscript{1}, Jenn Lewis\textsuperscript{1}, \&
Yoel Everett\textsuperscript{1}}
    \date{}
  
\shorttitle{Maternal Emotion Dysregulation and Child Outcomes}
\affiliation{
\vspace{0.5cm}
\textsuperscript{1} University of Oregon}
\keywords{emotion regulation, parenting, child outcomes\newline\indent Word count: X}
\usepackage{csquotes}
\usepackage{upgreek}
\captionsetup{font=singlespacing,justification=justified}

\usepackage{longtable}
\usepackage{lscape}
\usepackage{multirow}
\usepackage{tabularx}
\usepackage[flushleft]{threeparttable}
\usepackage{threeparttablex}

\newenvironment{lltable}{\begin{landscape}\begin{center}\begin{ThreePartTable}}{\end{ThreePartTable}\end{center}\end{landscape}}

\makeatletter
\newcommand\LastLTentrywidth{1em}
\newlength\longtablewidth
\setlength{\longtablewidth}{1in}
\newcommand{\getlongtablewidth}{\begingroup \ifcsname LT@\roman{LT@tables}\endcsname \global\longtablewidth=0pt \renewcommand{\LT@entry}[2]{\global\advance\longtablewidth by ##2\relax\gdef\LastLTentrywidth{##2}}\@nameuse{LT@\roman{LT@tables}} \fi \endgroup}


\DeclareDelayedFloatFlavor{ThreePartTable}{table}
\DeclareDelayedFloatFlavor{lltable}{table}
\DeclareDelayedFloatFlavor*{longtable}{table}
\makeatletter
\renewcommand{\efloat@iwrite}[1]{\immediate\expandafter\protected@write\csname efloat@post#1\endcsname{}}
\makeatother
\usepackage{lineno}

\linenumbers

\authornote{

Correspondence concerning this article should be addressed to Jackie
O'Brien, Postal address. E-mail:
\href{mailto:my@email.com}{\nolinkurl{my@email.com}}}

\abstract{
One or two sentences providing a \textbf{basic introduction} to the
field, comprehensible to a scientist in any discipline.

Two to three sentences of \textbf{more detailed background},
comprehensible to scientists in related disciplines.

One sentence clearly stating the \textbf{general problem} being
addressed by this particular study.

One sentence summarizing the main result (with the words ``\textbf{here
we show}'' or their equivalent).

Two or three sentences explaining what the \textbf{main result} reveals
in direct comparison to what was thought to be the case previously, or
how the main result adds to previous knowledge.

One or two sentences to put the results into a more \textbf{general
context}.

Two or three sentences to provide a \textbf{broader perspective},
readily comprehensible to a scientist in any discipline.


}

\begin{document}
\maketitle

\begin{verbatim}
## Observations: 97
## Variables: 6
## $ family_id      <dbl> 1001, 1002, 1003, 1004, 1005, 1006, 1007, 1008,...
## $ cbcl_int       <dbl> 10, 4, 15, 9, 10, 10, 5, 4, 3, 6, 3, 10, 13, 5,...
## $ cbcl_ext       <dbl> 13, 12, 20, 14, 18, 16, 7, 12, 3, 6, 0, 7, 17, ...
## $ ders           <dbl> 54, 59, 87, 75, 48, 65, 55, 53, 54, 48, 40, 68,...
## $ child_baseline <dbl> 7.038787, 5.819146, NA, 5.684124, NA, NA, 6.111...
## $ child_lego     <dbl> 5.952458, 5.132448, 6.669899, 4.372479, 5.04177...
\end{verbatim}

\begin{verbatim}
## Observations: 136
## Variables: 6
## $ family_id      <dbl> 1001, 1002, 1003, 1004, 1005, 1006, 1007, 1008,...
## $ ders           <dbl> 54, 59, 87, 75, 48, 65, 55, 53, 54, 48, 40, 68,...
## $ child_baseline <dbl> 7.038787, 5.819146, NA, 5.684124, NA, NA, 6.111...
## $ child_lego     <dbl> 5.952458, 5.132448, 6.669899, 4.372479, 5.04177...
## $ cbcl_subtype   <chr> "int", "int", "int", "int", "int", "int", "int"...
## $ cbcl_score     <dbl> 10, 4, 15, 9, 10, 10, 5, 4, 3, 6, 3, 10, 13, 5,...
\end{verbatim}

\begin{verbatim}
## Observations: 136
## Variables: 10
## $ family_id         <dbl> 1001, 1002, 1003, 1004, 1005, 1006, 1007, 10...
## $ ders              <dbl> 54, 59, 87, 75, 48, 65, 55, 53, 54, 48, 40, ...
## $ child_baseline    <dbl> 7.038787, 5.819146, NA, 5.684124, NA, NA, 6....
## $ child_lego        <dbl> 5.952458, 5.132448, 6.669899, 4.372479, 5.04...
## $ cbcl_subtype      <chr> "int", "int", "int", "int", "int", "int", "i...
## $ cbcl_score        <dbl> 10, 4, 15, 9, 10, 10, 5, 4, 3, 6, 3, 10, 13,...
## $ reactivity        <dbl> -1.086328857, -0.686697786, NA, -1.311645429...
## $ ders_c            <dbl> -16.102941, -11.102941, 16.897059, 4.897059,...
## $ reactivity_c      <dbl> 0.014032141, 0.413663212, NA, -0.211284431, ...
## $ ders_x_reactivity <dbl> -0.22595873, -4.59287831, NA, -1.03467229, N...
\end{verbatim}

\section{Results}\label{results}

\begin{verbatim}
## Warning: Removed 4 rows containing non-finite values (stat_smooth).
\end{verbatim}

\begin{verbatim}
## Warning: Removed 4 rows containing missing values (geom_point).
\end{verbatim}

\begin{figure}
\centering
\includegraphics{DataPrepScript_apa_style_files/figure-latex/descriptives-1.pdf}
\caption{}
\end{figure}

\begin{verbatim}
## 'data.frame':    136 obs. of  10 variables:
##  $ family_id        : num  1001 1002 1003 1004 1005 ...
##  $ ders             : num  54 59 87 75 48 65 55 53 54 48 ...
##  $ child_baseline   : num  7.04 5.82 NA 5.68 NA ...
##  $ child_lego       : num  5.95 5.13 6.67 4.37 5.04 ...
##  $ cbcl_subtype     : chr  "int" "int" "int" "int" ...
##  $ cbcl_score       : num  10 4 15 9 10 10 5 4 3 6 ...
##  $ reactivity       : num  -1.086 -0.687 NA -1.312 NA ...
##  $ ders_c           : num  -16.1 -11.1 16.9 4.9 -22.1 ...
##  $ reactivity_c     : num  0.014 0.414 NA -0.211 NA ...
##  $ ders_x_reactivity: num  -0.226 -4.593 NA -1.035 NA ...
\end{verbatim}

\begin{tabular}{rrrr}
\toprule
DERS\_mean & DERS\_SD & Reactivity\_mean & Reactivity\_SD\\
\midrule
70.10294 & 22.33027 & -1.100361 & 0.6520285\\
\bottomrule
\end{tabular}

\begin{tabular}{lrr}
\toprule
cbcl\_subtype & cbcl\_mean & cbcl\_SD\\
\midrule
ext & 16.28333 & 9.492662\\
int & 11.17318 & 7.539277\\
\bottomrule
\end{tabular}

\section{Methods}\label{methods}

We report how we determined our sample size, all data exclusions (if
any), all manipulations, and all measures in the study.

\subsection{Participants}\label{participants}

\subsection{Material}\label{material}

\subsection{Procedure}\label{procedure}

\subsection{Data analysis}\label{data-analysis}

We used R (Version 3.5.1; R Core Team, 2018) for all our analyses.

\section{Discussion}\label{discussion}

\newpage

\section{References}\label{references}

\begingroup
\setlength{\parindent}{-0.5in} \setlength{\leftskip}{0.5in}

\hypertarget{refs}{}
\hypertarget{ref-R-base}{}
R Core Team. (2018). \emph{R: A language and environment for statistical
computing}. Vienna, Austria: R Foundation for Statistical Computing.
Retrieved from \url{https://www.R-project.org/}

\endgroup


\end{document}
